\section{Conclusion}
\label{sec:conclusion}

In this paper, we have presented a comprehensive survey of recent developments in LLM-driven phone automation technologies, illustrating how large language models can catalyze a paradigm shift from static script-based approaches to dynamic, intelligent systems capable of perceiving, reasoning about, and operating on mobile GUIs. We examined a variety of frameworks, including single-agent architectures, multi-agent collaborations, and plan-then-act pipelines, demonstrating how each approach addresses specific challenges in task complexity, adaptability, and scalability. In parallel, we analyzed both prompt engineering and training-based techniques (such as supervised fine-tuning and reinforcement learning), underscoring their roles in bridging user intent and device action.

Beyond clarifying these technical foundations, we also spotlighted emerging research directions and provided a critical appraisal of persistent obstacles. These include ensuring robust dataset coverage, optimizing LLM deployments under resource constraints, meeting real-world demand for user-centric personalization, and maintaining security and reliability in sensitive applications. We further emphasized the need for standardized benchmarks, proposing consistent metrics and evaluation protocols to fairly compare and advance competing designs.

Looking ahead, ongoing refinements in model architectures, on-device inference strategies, and multimodal data integration point to an exciting expansion of what LLM-based phone GUI agents can achieve. We anticipate that future endeavors will see the convergence of broader AI paradigms—such as embodied AI and AGI—into phone automation, thereby enabling agents to handle increasingly complex tasks with minimal human oversight. Overall, this survey not only unifies existing strands of research but also offers a roadmap for leveraging the full potential of large language models in phone GUI automation, guiding researchers toward robust, user-friendly, and secure solutions that can adapt to the evolving needs of mobile ecosystems.
